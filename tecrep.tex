%%%
%% v3.3 [2020/05/14]
%\documentclass[Proof,technicalreport]{ieicej}
\documentclass[technicalreport]{ieicej}
%\usepackage{graphicx}
%\usepackage[fleqn]{amsmath}
%\usepackage{amsthm}
%\usepackage{amssymb}
\usepackage{latexsym}
\usepackage{newtxtext}
\usepackage[varg]{newtxmath}

\def\IEICEJcls{\texttt{ieicej.cls}}
\def\IEICEJver{3.2}
\newcommand{\AmSLaTeX}{%
 $\mathcal A$\lower.4ex\hbox{$\!\mathcal M\!$}$\mathcal S$-\LaTeX}
%\newcommand{\PS}{{\scshape Post\-Script}}
\def\BibTeX{{\rmfamily B\kern-.05em{\scshape i\kern-.025em b}\kern-.08em
 T\kern-.1667em\lower.7ex\hbox{E}\kern-.125em X}}

\jtitle{Yoloを用いた画像認識による交差点付近におけるリアルタイム車両状態推定}
\etitle{A Real-time Estimation of Vehicle Status Near an Intersection Using \\ Image Recognition by Yolo}
\authorlist{%
 \authorentry[anzy@cps.im.dendai.ac.jp]{安齋凌介}{Ryosuke ANZAI}{DTokyo}% 
 \authorentry[mito@iis.u-tokyo.ac.jp]{伊藤昌毅}{Masaki ITO}{UTokyo}% 
 \authorentry[takog@iis.u-tokyo.ac.jp]{大口敬}{Kei OGUCHI}{UTokyo}% 
 \authorentry[iwai@cps.im.dendai.ac.jp]{岩井将行}{Masayuki IWAI}{DTokyo}% 
}
\affiliate[DTokyo]{東京電機大学大学院未来科学研究科情報メディア学専攻\hskip1zw
  〒120--8551 東京都足立区千住旭町5}
 {Department of Information Systems and Multimedia Design,
  Tokyo Denki University\hskip1em
  SenjuAsahi-cho 5, Adachi-ku, Tokyo,
  120--8851 Japan}
\affiliate[UTokyo]{東京大学生産技術研究所\hskip1zw
  〒153--8505 東京都目黒区駒場4--6--1}
 {Institute of Industrial Science, the University of Tokyo\hskip1em
  4--5--6 Komaba, Meguro-ku,
  153--8505 Japan}

%\MailAddress{$\dagger$hanako@denshi.ac.jp,
% $\dagger\dagger$\{taro,jiro\}@jouhou.co.jp}

\begin{document}
\begin{jabstract}
  交通工学はエッジセンサから得られたデータを活用することで,
  可能性が大きく広がる分野である.速度違反検知,ナンバープレート検知,渋滞検知などは日本のみならず先進国で導入されている.
  しかしながら,それらのデータを交通の最適化に活用している例はまだ少ない.
  これは,データの質の低さ,ラベルの欠如,データの利権などが障壁になっていることが考えらえる.
  またそれらのセンサは大規模かつ高額になることが多く,交通量の多い交差点や高速道路に限定して設置されていることが多い.
  したがって,我々が普段使う道路や交差点ではこれらの導入は遅れていることが現状である.そこで,我々はオープンソースデータと安価なエッジセンサを活用し,
  交差点付近の車両の状態を推定することができるデバイスを開発した.
  これにより今まで取ることができなかったその地点の詳細な車両のデータを小規模で安価に取ることが可能となる.
\end{jabstract}
\begin{jkeyword}
  Yolo, エッジセンサ
\end{jkeyword}
\begin{eabstract}
  Traffic engineering is a field where the possibilities expand greatly by utilizing data obtained from edge sensors. Speed violation detection, license plate detection, traffic jam detection, etc. have been introduced not only in Japan but also in other developed countries.
  However, there are still few examples of the use of such data for traffic optimization.
  This may be due to the low quality of the data, lack of labels, and data rights.
  In addition, these sensors are often large and expensive, and are often installed only at busy intersections or highways.
  To solve this problem, we have developed a system using open source data and inexpensive edge sensors.
  We have developed a device that can estimate the state of vehicles near an intersection using open source data and inexpensive edge sensors.
  which makes it possible to obtain detailed vehicle data on a small scale at a low cost.
\end{eabstract}
\begin{ekeyword}
  Yolo, Edge sensors
\end{ekeyword}
\maketitle

\section{まえがき}
IoT(Internet of Things)の普及に伴い,さまざまな場所でのセンシングやデータの収集が可能となった.
交通工学の分野でも例外ではなく,古くは車両データを扱った技術として速度違反自動取締装置(オービス)が挙げられる.
また,近年ではITS(Intelligent Transport Systems:高度道路交通システム)
の観点からさらに高度で詳細な交通データを扱う研究やプロジェクトが注目されている.
IEEE/CVF Conference on Computer Vision and Pattern Recognition (CVPR) Workshopsでは
AI City Challenge\cite{Naphade_2020_CVPR_Workshops}と称し,ITSに関するワークショップを毎年開催している.
ここでは,カメラによる車両の識別,スピード検知,ナンバープレートの読み込みによるマルチカメラでの車両の再識別などタスクが用意されており,
世界的に見てもITSが注目されていることがわかる.

しかし,現状としてそれらの装置が一般的な道路や交差点に取り付けられていることは少なく,高速道路や交通量の多い幹線道路に限定し取り付けれていることが多い.
これは,装置自体が高価で大規模化しやすいことが原因であると考えられる.さらに,日本においてはそれらのデータは,違反や犯罪,渋滞の検知として用いられていることが多く,
交通の最適化という観点からデータを活用している事例は少ない.これは,データの質の低さ,ラベルの欠如,データの利権などが障壁になっていることが考えられる.
したがって,交差点や道路上の車両の状態を推定するには,IoTデバイスの開発を公開されているデータや安価なエッジセンサで行う必要がある.
これらのことから本研究では,2次元のカメラ映像から交差点付近にある複数台の車両の追跡をリアルタイムで行うデバイスの開発を目標とした.

こうしたカメラ画像から複数のオブジェクトを検出し追跡するタスクをコンピュータビジョンや機械学習分野ではMOT(Multi Object Tracking)と呼ぶ.
MOTでは,リアルタイムか否かで大きく二つにタスクを分けることができる.
リアルタイムではない処理としてTrackletNet Tracker(TNT)\cite{wang2019exploit}は,2次元のカメラ映像からオブジェクトを検出し,オブジェクトの軌跡を深層学習
を用いて推定していく方法である.TNTは,カメラ映像内で複雑かつ大量のオブジェクトを追跡することに向いているものの,カメラ映像を全て読み込んで処理するため,リアルタイム性
はなく,本研究では用いることができない.
一方でリアルタイムでの処理として,Simple Online and Realtime Tracking(SORT)\cite{bewley2016simple}がある.これは,検出されたオブジェクトの座標を
カルマンフィルタやハンガリアンアルゴリズムを用いて,リアルタイムでのオブジェクトの追跡を可能としている.
前述で述べたTNTと比べると複雑なMOTを行うことは難しいものの,ある程度規則性を持った車両などのMOTはリアルタイムで処理することが可能である.

可能にした方法であり,カルマンフィルタ



\bibliography{bibsample} %bibファイルの.bibの前の部分
\bibliographystyle{junsrt} %引用された順番に出力

\end{document}
